\section{Results} \label{sec:result}

Using specification \ref{incalone} with the Google Trends outcomes replicates the results of \cite{bh1}.\footnote{Note that we substitute $Internet$ or $Computer$ for $Income$ when necessary to get the appropriate results.} The results are shown in Table \ref{table:rep}. Panel (A) is an exact replication of their DMA-level regressions from the county-level. To ensure perfect agreement, the DMA-level statistics from the ACS in 2018 (the same as used in their paper) are brought to the county-level. For all statistics except population, this procedure assigns the DMA value to the counties within the DMA. To ensure the correct DMA population in the weighting process, we divvy up the DMA population amongst the counties by assigning each county a population equal to the DMA population divided by the number of counties in the DMA.
Panel (B) shows the results of the same regressions with the county-level statistics from the ACS in 2019. The magnitudes of the effects are smaller. The variation at the county-level reduces power of county-level to predict the DMA-level outcome. Given that counties are a more fine-grained region than MSAs, we therefore expect the county-level estimates, such as the coefficient $\beta_{inc}$, to be more biased than the estimate for MSA level statistics, such as $\beta_{tele}$. I present several different approaches to the regressions below, which highlight how the results can change when we take different approaches to the problem of dealing with the data’s differing geographical levels. Fortunately, this is not an issue for regressions which use Zearn outcomes. \par
	For the featured regressions in this study, whose results are shown in \ref{table:fullreg}, the four outcomes are search interest for school-centered resources, search interest for parent-centered resources, Zearn engagement, and Zearn badges. I restrict the dataset to counties within MSAs to ensure that systematically missing teleworkability data does not bias estimates. I drop the weeks of March 1, March 8, March 15, and March 23 since schools were in the process of closing during this time.\footnote{Figures \ref{fig:g_time20}-\ref{fig:z_inc_tele} support dropping this adjustment period, as both the outcome variables and the income and teleworkability effects changed most rapidly during early March.} For each of the outcomes, I perform a regression with specifications \ref{incalone}, \ref{inctele}, and \ref{inctelecontrols}. In comparison to the model that only considers income as an explantory variable, we find that the income-interaction coefficients decrease when teleworkability controls are added. Additionally, the teleworkability coefficients do not change significantly when the additional controls are included in the model. Therefore, the effect of teleworkability is not due to its correlation with the fraction of households that have computers or broadband internet, which are other measures of SES used \cite{bh1}.\footnote{Correlations: $Computer$ and $Teleworkability$: r=0.447; $Internet$ and $Teleworkability$: r=0.3534; $Computer$ and $Income$: r=0.7825; $Internet$ and $Income$ = 0.7065; $Computer$ and $Internet$: r=0.9026.}
	The reduction in the income-interaction coefficients is strongest for the Zearn outcome variables. As suggested by $\frac{\beta_{inc\left(2\right)}-\beta_{inc\left(4\right)}}{\beta_{inc\left(2\right)}}$, roughly 60\% of the income effect for Zearn engagement and 48\% of the effect for Zearn achievement is explained by teleworkability.  Additionally, with the introduction of the other SES-related controls, the income coefficients diminish even more, while the teleworkability coefficients remain unchanged. A 5.5\% (one standard deviation) increase in teleworkability increases the expected change in Zearn engagement and Zearn achievement due to COVID by 6-7\%.  \par
	The effect is even larger for search interest for school-centered resources, suggesting roughly 10\% increase in the change in search interest relative to March 1 for a one standard deviation increase in teleworkability. However, stark changes when we use different methodological approaches raises the level of uncertainty associated with the Google Trends outcomes results. Table \ref{table:methodcomparison} displays results for the same specification with slightly different approaches to the geographic levels. In the main approach displayed in \ref{table:fullreg}, we include low-search interest areas and crosswalk the DMA-level trend outcomes and MSA-level teleworkability percentages to the county-level (with values set equal to the minimum possible search interest, as discussed above). However, we could instead drop low-search interest areas or aggregate the county-level statistics to the DMA-level. All four possibilities are shown in the table. \par
We see that aggregating to the DMA-level decreases the teleworkability intereaction coefficient and $\frac{\beta_{inc\left(2\right)}-\beta_{inc\left(4\right)}}{\beta_{inc\left(2\right)}}$. Excluding low-search interest has the reverse effect but is not as large. The decrease in effect size is especially stark for parent-centered resources, which show that the introduction of teleworkability as a control \textit{increases} the income effect ($\frac{\beta_{inc\left(2\right)}-\beta_{inc\left(4\right)}}{\beta_{inc\left(2\right)}}$ is negative). Given that teleworkability is twice removed from its true level (once by crosswalking to the county level, and again by aggregating the county-level estimate to the DMA-level), it is likely that its power as a predictor has been diminished. Regardless of the approach, a higher teleworkability percentages predicts a larger increase in search interest at COVID for school-centered resources than for parent-centered resources. A lower bound estimate for the impact of a 5.5\% increase in teleworkability is a 0.498\% increase in search interest for school-centered resources after COVID. Similarly, the income effect reduces by at least 22.4\%. Given the abrupt difference in coefficients by approach for parent-centered resources, there is not enough evidence to suggest that teleworkability contributed to the disproportionate increase in search interest among high-income groups due to the pandemic. \par
	Figures \ref{fig:g_time20} - \ref{fig:z_inc_tele} display the coefficients of interest from regressions with specifications \ref{week}, \ref{inconlyweek}, and \ref{incteleweek} for the same outcome variables. It illuminates how the general effects described above emerged on a weekly basis. Figures \ref{fig:g_time20} and \ref{fig:z_time20} demonstrate that search interest increased abruptly while schools began to close before slowly declining towards its typical level as summer approached. Conversely, the usage of Zearn dropped starkly in early March, and then fluctuated around its usual level at the end of the school year. Figures \ref{fig:g_inc_only} and \ref{fig:z_inc_only} confirm that teleworkability plays a more important explanatory role in the income gaps observed in the Zearn usage data. The general shape to the time-pattern of the coefficients does not change with the addition of teleworkability controls for any outcome variable.. However, when the income effects undergo their most rapid change, we observe the largest difference between the model with and without teleworkability controls. For example, the largest gap between the income coefficients in the two models for the Zearn results occurs in early April, when the income effect diminishes much more drastically for the model with teleworkability controls. We notice a similar effect for search interest for school-centered resources. \par
	Figures \ref{fig:g_inc_tele} and \ref{fig:z_inc_tele} allow us to compare how the relative importance of income and teleworkability as a predictor of search interest changes over time during the 2020 school year. For seach interest for school-centered resources, teleworkability is uniquely important in late March and early April, as it rises to over a 10\% increase in search interest per 5.5\% increase in teleworkability while the income effect falls to zero. Following this point, income and teleworkability are roughly of equivalent importance, with a late trend suggesting that income becomes more important in the weeks right before summer. The equivalent coefficients for parent-centered resources are more stable, but still exhibit a behavior that is worth noting. Teleworkability and income seem to be of roughly equivalent importance until the middle of April, when the teleworkability coefficients temporarily plummet for two weeks. For the remainder of the school year, the effect of a unit standardized increase in income on search interest for parent-centered resources was larger than the effect of a unit standardized increase in teleworkability. \par
 	The importance of teleworkability relative to income quickly rises during March 2020 for the Zearn outcome measures. Figures \ref{fig:engagement_time_all} and \ref{fig:badges_time_all} put this increase in perspective with general seasonal patterns. It displays the time trend of the interaction coefficients for a regression over the 2019 and 2020 calendar years. Based on these two years, the relative importance of teleworkability jumps during the summer, falls during the middle of the sumer, and then drops at the beginning of the school year. COVID changed this pattern by shifting the sharp increase that typically happens at the beginning of the summer to the beginning of March. When summer did arrive, there was an additional boost, but not as large in magnitude as the increase in the summer of 2019. Before COVID, the income coefficients remained roughly consistent, fluctuating about zero (meaning that the expected change in engagement per increase in income is roughly the same as it is in March 1, 2020) and exhibiting a slight bump at the beginning of summer. However, the importance of income spikes during March 2020. It persists for the remainder of the school year, and diminishes over the course of the summer. Unlike the prior fall, the importance of income returns during the fall of 2020, whereas the magnitude of the teleworkability coefficients are roughly equal to what they were the previous year. \par
In all, the results of these regressions confirm that education gaps widened at the onset of the COVID-19 pandemic, suggest that part of the effect was due to differences in the ability to work from home across regions, and demonstrate that both teleworkability and income both were contributing factors to the phenomenon. These statements hold for both differences in search patterns for online resources and in changes in student use of online learning platforms. \par

