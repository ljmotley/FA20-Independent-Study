\section{Conclusion} \label{sec:conclusion}
\if\tenpg1
{\small
    \textit{Note: This section has been abridged to meet the page requirements. Link to full paper on the first page.}
}
\fi

This paper demonstrates the need to consider remote work when interpreting the inequalities in education that emerged at the onset of the COVID-19 pandemic, adding to the growing literature on remote work and school.
The growth in inequality in educational outcomes observed by \cite{bh1} and \cite{chetty} drops dramatically when teleworkability controls are included.
I estimate that more than 20\% of the relative increase in search interest for educational resources and up to 50\% of the increase in student engagement and achievement on Zearn that is attributed to income can be explained by teleworkability.
A one percent increase in county-level teleworkable jobs has a comparable or larger positive effect on student and parent behavior than an equivalent increase in income.
The effect is largest for student achievement on the math platform Zearn.

These findings hold when using other measures of teleworkability than \cite{dingel}, such as work from home scores constructed using \cite{mongey}.
They also pass placebo-checks using other correlates with median household income, such as
the share of households with computers or broadband internet.

This work motivates finer-grained studies to identify the underlying mechanisms for the COVID-induced educational inequalities.
Additionally, it is an example of a side effect of a labor market shock leading to an unanticipated disparity.
The prevalence of these effects should be investigated, and the groups who are typically most impacted by them should be identified.
It is possible that factors related to ``adaptability'' (e.g., being able to work from home) are not given enough value in labor market decisions due to the unpredictability of events like COVID-19.

