\section{Conclusion} \label{sec:conclusion}

This research demonstrates the need to consider the impact of remote work when interpreting the inequalities in education that emerged at the onset of the COVID-19 pandemic. Using data from Google Trends and the Zearn online learning platform, it estimates that more than 20\% of the relative increase in search interest for school-centered resources and up to 50-60\% of the increase in student engagement and achievement on Zearn that is attributed to a standard deviation increase in median household income can be explained by teleworkability. One standard deviation increase in the percentage of teleworkable jobs (5.5\%) has a comparable or larger effect to a standard deviation increase in income (\$15,500) for all outcomes. The effect is largest for the Zearn outcomes. Additionally, when differences in the results when additional controls are added and the methodology is adjusted the claims for the Trends outcomes (especially parent-centered resources)  are weakened, while the Zearn outcomes are strengthened. For engagement and achievement on Zearn, as well as search interest for school-centered online learning terms, our results predict a 5-10\% increase in the outcome measures for a standardized increase in income or teleworkability post-COVID. Investigating the impact at a weekly-level reveals that a spike importance of teleworkability in increasing the Zearn outcomes likely would have occurred in the summer of  2020 regardless, but that the pandemic shifted the jump to early March and increased its magnitude. These results motivate household level studies that would identify the most differences in educational environments that caused the observed inequalities.

