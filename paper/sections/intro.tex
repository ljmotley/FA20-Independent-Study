\section{Introduction} \label{sec:introduction}
\if\tenpg1
{\small
    \textit{Note: This section has been abridged to meet the page requirements. Link to full paper on the first page.}
}
\fi

In early March 2020, COVID-19 abruptly interrupted the daily routines of both children and adults.
From March 16-March 24, every state ordered or recommended school closures and
all except Montana and Wyoming ultimately extended the guideline to the end of the
year.\footnote{School closure information from Education Week.}
Early studies suggest that the shift to remote learning during COVID-19 will disproportionately impact low-income students,
an effect that has also been observed during previous school closures
\citep{vonHippel, horowitz, jaeger, malkus}.
Unlike in past closures, however, COVID-19 caused workers across the country to begin working from home
at the same time that schools sent children home.
During the first week of April, over one-third of workers reported having shifted
to remote work in response to the crisis \citep{bynjolfsson}.
This paper examines how these two responses interacted, showing that parents
ability to work from home impacted educational outcomes at the onset of the pandemic.

The known connections between income, teleworkability, and student achievement motivate this investigation.\footnote{Throughout the paper, I refer to \textit{teleworkability} as the capability for a job to be done at home.}
First, higher-wage jobs are more teleworkable
\citep{bartik, dingel}.
Thus, at the time of the school closures, income
correlated with both a higher likelihood of the worker shifting to remote work and a lower severity of negative educational outcomes for the student.
Moreover, \cite{bettinger} provide evidence that stay-at-home parenting has a positive impact on achievement of their students,
and \cite{sonnemann} document that low-socioeconomic status (SES) students are less likely to have parent assistance and support with school.
Together, these facts suggests that parents’ working remotely may have a causal impact on the educational outcomes of students through at-home parent support.
While this mechanism would be fascinating to explore with richer, individual-level data,
this paper simply suggests at this connection through regional-level analyses.

Specifically, I use the \cite{dingel} classification of feasibility of working at home to explore the impact of teleworkability on COVID-induced educational inequalities in two datasets.
The first is Google Trends search data.
\cite{bh1} show that high-income areas exhibited significantly larger
increases in the seeking of online resources after COVID as measured by Google searches.
They also group the words as ``parent-centered'' and ``school-centered'' to indicate who the nature of the search.
I pair this proxy for adult engagement with a direct measure of child behavior on an online math platform called Zearn.
\cite{chetty} demonstrate that high-income areas showed substantially smaller decreases in participation and achievement on this website.\footnote{The geographical patterns of these inequalities fit the spatial variation in income and teleworkability (see appendix), further motivating the study).}
\if\tenpg0
Every analysis in the paper uses these four outcomes (with the measure in parentheses): engagement in the child's online learning (Google Trends search interest for school-centered resources), parent engagement in the child's online learning (Google Trends search interest for parent-centered resources), student engagement in online learning (Zearn engagement), and student achievement in online learning (Zearn badges).
At the regional level, these measures are aggregates.
Therefore, Google search interest is an aggregate measure of effort to supplement or replace school with online resources, and the Zearn statistics are aggregate estimates of the actual use and achievement of students on online learning platforms, respectively.
\fi

Using a difference-in-differences design, I find that high-income areas saw a significant relative increase in income inequality for each of the Zearn and Trends outcomes after the pandemic for the remainder of the 2020 school year.
However, the inequality associated with an increase in income diminishes after including controls for teleworkability.
In fact, an increase in the percentage of teleworkable jobs has a comparable or larger effect on each outcome as an equivalent increase in median household income.
Moreover, the same cannot be said when the exercise is repeated with other correlates of income, such as the share of households with computers or broadband internet.

\if\tenpg0
I also examine a weekly event study of these results.
There is no evidence of pre-trends.
Additionally, the importance of teleworkability for Google search interest spikes in early March, while and the Zearn lag behind by two to three weeks.
This discrepancy is likely due to the fact that schools did not immediately start ``distance learning''.
Thus, the gap reflects the time that parents and schools searched for resources while students had time off due to the closure.
Observing the long-term patterns of this event study for the Zearn outcomes reveals that a jump in the importance of teleworkability occurs at the beginning of the summer of 2019.
This seasonality makes sense.
It is natural that parents who work from home have a larger impact on their child's online learning engagement and achievement during the summer, when their child is also at home.
The pandemic shifted this seasonal pattern into early March in 2020, and the teleworkability coefficients increased to a much lesser degree at the start of the summer in 2020.
\fi

Overall, this paper demonstrates that teleworkability is a potential mechanism driving the observed inequalities at the onset of the COVID-19 pandemic.
The most convincing effects are for measures of actual student engagement and the seeking of specific, school-related resources rather than general, parent-related resources.
While better identification is necessary to be confident about the causal effect of teleworkability on online learning, these results motivate further work with richer, individual level data.
By better understanding the mechanisms that create inequalities, we can craft reactive policies in the present and respond more proactively to similar shocks in the future.
If teleworkability plays a crucial causal role in online-learning educational inequalities, we would expect purely monetary relief to be less successful at addressing these gaps.
Instead, children would need direct educational assistance.

Moreover, this work suggests that idiosyncratic features of a labor market shock may cause unexpected unanticipated disparities to emerge.
In this case, social distancing may have led workers who cannot work from home to be disadvantaged.
More research should be done to identify the groups that are most susceptible to unpredictable negative side effects of these shocks due to variation in adaptability.
%I posit that---like with teleworkability during a pandemic---the most susceptible people will be the least privileged.

