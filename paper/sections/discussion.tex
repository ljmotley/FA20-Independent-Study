\section{Evaluating the identification strategy \& \\ discussing broader implications} \label{sec:discussion}
\if\tenpg1
{\small
    \textit{Note: This section has been abridged to meet the page requirements. Link to full paper on the first page.}
}
\fi

\if\tenpg0
Teleworkability could be an important explanatory factor as we research and respond to inequalities caused by COVID.
I find larger increase in searches for online learning resources for areas with higher teleworkability, especially for school-centered resources.
\cite{chetty} observed that high-income areas saw less of a decrease in the both overall use and achievement on the Zearn platform.
The results suggest a significant share (as high as 50\%) of the gap could be explained by differences in the ability to work from home.
%For the remainder of the 2020 school year, a predict roughly a 6-7\% increase in the effect of a one standard deviation (5.5\%) increase in teleworkability percentage on the Zearn outcomes compared to a 2-4\% increase in the effect of a one standard deviations (\$15,500) increase in median household income. \par

%        It is plausible that teleworkability correlates with other important factors, such as unemployment, biasing our coefficients.

The overall time trends add to the picture. During 2019 and 2020, the relative importance of teleworkability exhibits seasonal behavior. It drops at the onset of the school year, increases over the course of the year, and then spikes up during the summer. The pattern is interrupted by COVID-19, which appears to cause the summer spike in the importance of teleworkability to occur earlier, when schools closed due to the pandemic. The capability to work at home could be the causal component to both the spike in the summer and its shift to the spring in 2020: when kids are at home, they are more likely to do their work if a parent is also there.

However, there are other possible explanations. For example, it could be that motivated people are more likely to be the type of people who seek out teleworkable jobs. A motivated person may also dedicate more time and energy to their children’s education when they are not in school, which could explain the observed effects. \par
\fi
The framework and results of this research should be considered when crafting policy to address the emerging educational inequalities. There are two possible explanations for what happened at the arrival of COVID. One, there was an immediate shock and high-income households responded better to it, but the slopes of educational progress over time for high-income groups and low-income groups did not permanently change relative to each other. Two, COVID induced a permanent change in the educational environment that caused the slope of educational progress over time for high-income groups to increase relative to the slope of low-income groups educational progress over time. The former warrants short-term monetary relief to close a temporary gap. The latter requires a lasting solution. \par
It remains to be seen which explanation is true. But, given that teleworkability may be an important causal piece and is expected to continue even after the pandemic, we may need to consider a long-term approach to the problem. If the problem is not purely monetary, money alone will not solve it.  Getting a clearer picture of the socioeconomic factors causing the gaps and their relative importance is an important starting point. This goal can be accomplished with regional-level data as used in this paper.

The next step is to determine how exactly educational environments changed. For example, is teleworkability important simply because they struggle with online learning technology and need adult assistance, or is it important because adults take a generally more active role in their children’s lives when they work from home? Is it a signal for something else entirely, such as personality differences as described above? Policies may be effective or ineffective based on the answers to these questions. We will need research at the level of the household to fully understand the cause of these trends and combat the inequalities that emerged due to them. And, armed with these answers, we may be able to tackle future disruptions more proactively.

