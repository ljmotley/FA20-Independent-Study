At the onset of COVID, the widening of educational achievement gaps along income was observed both using Google search data for online schooling resources and in student engagement on the online learning platform Zearn.
This paper identifies teleworkability as a potential mechanism.
It uses median household income and the percentage of teleworkable jobs in an area as independent variables.
At the county-level, it regresses both income-alone and income and teleworkability on the Trends and Zearn outcome measures.
The effect of a standard deviation (\$14,000 per year in 2018) increase in income on the outcome measures post-COVID diminishes by roughly 40\% when teleworkability controls are added.
The effect of a standard deviation increase in the percentage of teleworkable jobs (5.6\%) is comparable to the equivalent income effect.
Both increase the Zearn outcomes and searches for parent-centered resources by 5-10\% post-COVID, and increase searches for school-centered resources by more than 10\%.
I intend to make some slight adjustments and add additional controls to the model to better estimate the causal role of teleworkability.
Given that teleworkability likely plays a role in these inequalities and will continue in some capacity indefinitely, long-term solutions may be required.
