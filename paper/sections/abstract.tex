I identify parents' ability to work from home (teleworkability) as a potential mechanism for the widening income inequality in children's educational achievement due to COVID-19.
I use county-level difference-in-differences regressions to estimate the change in the impact of teleworkability on educational outcomes related to both adult-support and child-behavior---measured using Google Trends search interest and student engagement on an online math platform, respectively---immediately after March 1, 2020.
Not only is this change significant and positive for all outcomes, the documented COVID-induced income-inequalities in these measures diminish after adding teleworkability controls.
Moreover, these results pass placebo-checks: I substitute other correlates of income with teleworkability, such as the share of households with a computer, and find no effect.
I also observe seasonality in the relative importance of teleworkability on the student-behavior outcomes; specifically, COVID-19 shifted forward an annual summer spike in this teleworkability effect.
As children are on break in the summer, this pattern suggests that parent-student contact may be an important causal factor in the results.
Finer-grained work should be done to improve the identification of these effects and explore underlying mechanisms.
Given that the rate of teleworkability will likely stay above pre-COVID levels indefinitely, short-term monetary support to close these gaps may be ineffective.
Long-term solutions must incorporate direct educational support.
