This section introduces the idea of version control and applies it to a single-machine user.
(This is sufficient for applying version control to confidential-data settings where you only work on one server.)
A later section in the \textbf{Collaboration} block addresses using version control when sharing across machines.

\subsection{Introduction to version control}

Version control is the management of changes to code and documents.
It should be applied to all textual content in a research project.
Version control tools allow us to
\begin{itemize}
	\item Keep a complete history of our project without any additional effort or files
	\item Hit ``undo'' (at arbitrarily large scale) at will
	\item Collaboration: Edit code simultaneously without creating disastrous conflicts
	\item Collaboration: Easily track coauthors' contributions and changes
\end{itemize}
All real programmers use version control. The dominant version control software is Git.

To understand version control, please read the following short introductions:
\begin{itemize}
	\item Gentzkow \& Shapiro - ``\href{https://web.stanford.edu/~gentzkow/research/CodeAndData.pdf}{Chapter 3: Version Control}''
	\item Michael Stepner - ``\href{https://michaelstepner.com/blog/git-vs-dropbox}{git vs. Dropbox from a researcher's perspective}''
\end{itemize}

\subsection{Which files should be kept in version control?}

Version control is appropriate for any plaintext file.
Binaries (like PDFs) are less suitable, since you can't examine their contents.

Obviously all code should be in version control. What about writing?
\begin{itemize}
	\item \LaTeX\ is perfect; it's plaintext.
	\item LyX is suitable for version control, because \texttt{.lyx} files are plaintext.
	\item Jonathan prefers to err on the side of documenting and committing too much. There's little downside.
\end{itemize}

A file called \texttt{.gitignore} identifies paths that should not be tracked by Git.
\begin{itemize}
\item
We don't commit data.
I therefore typically put \texttt{tasks/*/output} in \texttt{.gitignore} so that the repository doesn't track output files.
\item
Git sees symbolic links as their targets.
I therefore put \texttt{tasks/*/input} in \texttt{.gitignore} so that the repository doesn't track data files.
\item
To produce logbook content (and the paper), you may want to commit tables and figures from these output folders.
To do so, you can always type, for example, \texttt{git add table1.tex -f}, where the \texttt{-f} force option causes the file to be tracked despite the gitignore rule.
\end{itemize}

\subsection{Version control in a single-machine environment}

To productively use git in a single-machine environment, you only need to know a half-dozen commands.
\begin{itemize}
\item \texttt{git init} initializes a repository in the current working directory
\item \texttt{git status} tells you what files have been modified since you last committed
\item \texttt{git diff} shows you those modifications (and can also show you the modifications between any two commits)
\item \texttt{git add} stages a file
\item \texttt{git commit} commits a set of changes
\item \texttt{git log} lists the history of commits
\end{itemize}
See ``\href{https://git-scm.com/book/en/v1/Git-Basics}{Git Basics}'' on these.

A couple more resources:
\begin{itemize}
\item For command-line Git, see Atlassian's \href{https://www.atlassian.com/git/tutorials/atlassian-git-cheatsheet}{Git cheatsheet}.
\item You may also want to glance at Jess Johnson - \href{http://www.grokcode.com/717/how-to-use-source-control-effectively/}{How To Use Source Control Effectively}.
\item If your University subscribes to Lynda, they offer \href{https://www.lynda.com/search?q=git}{a number of Git courses} (with which I have no experience).
\end{itemize}




Some notes on command-line Git by Ningyin Xu, a former RA, follow.

\subsubsection{Setting up a repository}

If you want to create a brand-new repo, you'll use the \texttt{git init} command.
\begin{lstlisting}[language=bash]
cd /path/to/your/existing/project
git init
\end{lstlisting}

\subsubsection{Saving changes to the repository}
Now that we have a repository, we can edit its contents and then commit changes to it.
The basic commands are:
\begin{lstlisting}[language=bash]
cd /path/to/repo
# make some changes to a file named "test.txt"
git add test.txt
git commit -m "edited test.txt"
\end{lstlisting}
This example introduced two additional git commands: \texttt{add} and \texttt{commit}.
The \texttt{git add} command adds a change in the working directory to the staging area.
It tells Git that you want to include updates to a particular file in the next commit.
However, \texttt{git add} doesn't really affect the repository in any significant way — changes are not actually recorded until you run \texttt{git commit}.
The \texttt{git commit} command commits the staged snapshot to the project history.
Committed snapshots can be thought of as ``saved'' versions of a project, Git will never change them unless you explicity ask it to.
In conjunction with these commands, you'll also need \texttt{git status} to view the state of the working directory and the staging area.
In general cases, you could do ``git status'' before ``git add'', and Git will tell you what files you have made changes and how you could deal with them.

Another common use case for \texttt{git add} is the \texttt{--all} option.
Executing \texttt{git add --all} will take any changed and untracked files in the repo and
add them to the repo and update the repo's working tree.
This may include hidden files.
You should list files that you do not want to be part of the repository in the file called \texttt{.gitignore} (\href{https://git-scm.com/docs/gitignore}{documentation}).


\subsubsection{Undoing changes}
The \texttt{git checkout} command is mostly used for its three functions:
checking out files, checking out commits, and checking out branches.
Checking out a commit makes the entire working directory match that commit.
This can be used to view an old state of your project without altering your current state in any way.
Checking out a file lets you see an old version of that particular file, leaving the rest of your working directory untouched.
Checking out branches will be discussed in the \textbf{Collaboration} section.
\\
\begin{lstlisting}[language=bash]
git checkout <commit> <file>
# Check out a previous version of a file. This turns the <file> that resides
# in the working directory into an exact copy of the one from <commit> and
# adds it to the staging area.

git checkout <commit>
# Update all files in the working directory to match the specified commit.
# But this is a read-only operation, the ``current'' state of your project
# remains untouched in the master branch.

git checkout master
# This lets you to get back to the "current" state (under the condition you
# don't have more than one branches).
\end{lstlisting}

In some case, you might want to undo your last commit.
To do that, use \texttt{git reset --soft HEAD\texttt{$\sim$}1}.
This command will remove the last commit from the current branch and keep previously commited files in the staged area.
For a more detailed description, see \href{https://www.atlassian.com/git/tutorials/undoing-changes/git-reset}{Atlassian guide}.
