These are Jonathan's standards and he's willing to argue for them.

\begin{itemize}
	\item Generally, follow Edward Tufte's \href{https://www.edwardtufte.com/tufte/books_vdqi}{\textit{Visual Display of Quantitative Information}}.
	Mostly important, maximize the \href{https://www.coursera.org/learn/python-plotting/lecture/qFnP9/graphical-heuristics-data-ink-ratio-edward-tufte}{data-ink ratio}.
	\item At minimum, let's adhere to Schwabish ``\href{https://www.aeaweb.org/articles?id=10.1257/jep.28.1.209}{An Economist's Guide to Visualizing Data}'' (\textit{JEP} 2014)
	\item Never make a graphic with fewer than a dozen data points. A dozen data points belong in a table.
	\item Please impose \texttt{graphregion(color(white))} on every \texttt{twoway} plot created in Stata.
\end{itemize}
