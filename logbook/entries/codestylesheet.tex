This section is for us to agree upon best practices in coding.
It's light at the moment.

\subsection{Language-free coding principles}

\subsubsection{Other people's principles}

\begin{itemize}
	\item See chapter 6, chapter 7, and appendix A of Gentzkow and Shapiro's \href{https://web.stanford.edu/~gentzkow/research/CodeAndData.pdf}{Code and Data for the Social Sciences: A Practitioner’s Guide}.
	\item QuantEcon's ``\href{https://julia.quantecon.org/getting_started_julia/introduction_to_types.html#A-Digression-on-Style-and-Naming}{A Digression on Style and Naming}''. Read their comments on the following:
	\begin{itemize}
	\item Code is read many more times than it is written
	\item Write code to be read in the future, not today
	\item Excess comments in code can make code harder to read
	\item Maintain the correspondence between the whiteboard math and the code
	\end{itemize}
\end{itemize}

\subsubsection{Do not hardcode values}

While I haven't fully embraced the \href{https://en.wikipedia.org/wiki/Functional_programming}{functional programming paradigm},
I do think that it is a mistake to use global variables or to hardcode values within functions.
See
Patrick Ball's \href{https://www.youtube.com/watch?v=ZSunU9GQdcI&t=36m43s}{argument}
and
page 28 of Gentzkow and Shapiro's \href{https://web.stanford.edu/~gentzkow/research/CodeAndData.pdf}{Code and Data}.

For example, sample selection decisions are arguments.
Constants do not belong inside functions.
Name them once, impose the value/criterion once, and then refer to that repeatedly within your function.
\href{http://yaml.org/}{YAML} is a good format for constants.

\subsection{Sharing code across machines}

We almost always want to run code on multiple machines.
Absolute file paths are problematic for shared code, because \texttt{/Users/jdingel/} is not a valid path for almost every computer on earth.
Here are two possible means of making shared code run on many machines:
\begin{enumerate}
	\item Set absolute file paths with conditional statements.
	\item Use only relative file paths.
\end{enumerate}

I strongly prefer the latter.

\subsubsection{Conditional absolute file paths}

Conditional absolute file paths exploit environmental variables to identify the machine and then set absolute file paths based on that.
For example, in Linux there is an environmental variable called \texttt{\$USER} at the shell command line.
Similarly, elements of Stata's c-class, such as \texttt{`c(pwd)'} and \texttt{`c(hostname)'}, can be used to identify a machine
and then run user-specific programs that set global variable to user-specific values.\footnote{
	Type \texttt{help creturn} in Stata for exhaustive documentation of the available arguments.
}
Here is a brief example in Stata.
\begin{lstlisting}[language=stata]
if "`c(os)'"=="Unix" & regexm("`c(hostname)'","midway")==1 {
	display "Running on RCC Midway server"
	global root "/project/xxx/xxxyyy/xyz_repo"
}
\end{lstlisting}

\subsubsection{Relative file paths}

Relative file paths work on any machine because they are relative.
In Jonathan's opinion, this is generally preferable.
It means that code can be cloned onto a new machine by any user and run immediately without having to write new user-specific code.
See \url{https://github.com/jdingel/DavisDingelMonrasMorales} for an example of a repository that only uses relative paths.

\href{https://help.github.com/articles/about-readmes/\#relative-links-and-image-paths-in-readme-files}{GitHub} agrees:
``Relative links are easier for users who clone your repository. Absolute links may not work in clones of your repository - we recommend using relative links to refer to other files within your repository.''

\subsubsection{Package management}

Package dependencies should be identified by \texttt{requirements.txt} (Python), \texttt{Manifest.toml} (Julia), and so forth.
Stata lacks a built-in package-dependency management tool.
These requirements should be committed to the repo so that anyone cloning the repo on a new machine has everything they need defined.
This is also \href{https://aeadataeditor.github.io/aea-de-guidance/preparing-for-data-deposit.html}{AEA journal policy}:
``packages/modules/etc. ... provide a setup program to install these (not manual instructions).''

\subsection{Limit lines to at most 100 characters}

Files should have lines no longer than 100 characters in length.
This is true both for code and \TeX.
The commands \texttt{diff} and \texttt{git diff} report lines that differ between files/commits.
Really long lines of code therefore can be a nightmare, since one may have to scroll quite far to spot a change.
This is really painful if a line is actually a full paragraph of text.
Writing each new sentence on a separate line makes it much easier to view changesets.

See \href{https://en.wikibooks.org/wiki/LaTeX/Collaborative_Writing_of_LaTeX_Documents\#Subversion_really_makes_the_difference}{Directives for using LaTeX with version control systems}
\begin{itemize}
\item Do not change line breaks without good reason.
\item Turn off automatic line wrapping of your LaTeX editor.
\item Start each new sentence in a new line.
\item Split long sentences into several lines so that each line has at most about 80 characters.
\item Verify that your code can be compiled flawlessly before committing your modifications to the repository.
\item Use diff to critically review your modifications before committing them to the repository.
\item Add a meaningful and descriptive comment when committing your modifications to the repository.
\end{itemize}
