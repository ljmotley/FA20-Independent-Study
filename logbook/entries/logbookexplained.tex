
This logbook is where we document the research project.
It can present preliminary results, sketch empirical specifications, etc.
There are separate chapters for research infrastructure, call notes, and research results.
These categories are easy to revise -- individual logbook entries should be modular pieces suitable for re-ordering.
Within chapters, entries are usually ordered chronologically. Entries are identified by date and author.
At their best, logbook entries presenting results can be directly pasted into the first draft of the paper.

\subsection{Committing logbook entries}

To add an entry this logbook:
\begin{enumerate}
	\item Write the logbook entry as a \TeX\ file. The \TeX\ file will be embedded within the logbook \TeX\ file, so do not include any \LaTeX\ preamble.
	\item Insert the entry into the logbook by editing \texttt{logbook.tex}.
	The logbook entry (section) title should include an informative title, the date, and author.
	Follow it by \texttt{\textbackslash input\{\}} with the filename.
	All other content is in the entries' individual \TeX\ files.
	\begin{lstlisting}[language=tex]
	\chapter{Call notes}
	\section{20210504 JD} \input{./entries/20210504Dingel.tex}
	\end{lstlisting}
\end{enumerate}

What not to commit
\begin{itemize}
	\item Please do not commit the logbook PDF to the repository.
	Doing so often creates merge conflicts when users preview their own logbook entries by compiling the logbook PDF locally.
	Similarly, do not commit the paper PDF to the repository.
	\item Do not commit the \texttt{logbook.aux}, \texttt{logbook.log}, nor \texttt{logbook.out} files generated by your \LaTeX compiler.
	Either add these to your \texttt{.gitignore} file or have your Makefile delete these after succesful compilation of the PDF.
\end{itemize}
